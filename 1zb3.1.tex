\documentclass[12pt]{article}
\usepackage[letterpaper, margin=0.5in]{geometry}
\usepackage{amsmath}
\DeclareMathSizes{12}{14}{10}{10}

\begin{document}


\section{Limits}


\subsection{$e$}

The function $e$ is defined as a continuous, differentiable function
$f(x)$ that satisfies $f'(x) = f(x)$ for all $x$ and $f(0) = 1$.\\
\\
\begin{displaymath}
    e = \displaystyle\lim_{n \to 0} \left(1 + n\right)^{\frac{1}{n}}
\end{displaymath}
\\
\begin{displaymath}
    e = \lim_{n \to \infty} \left(1 + \frac{1}{n}\right)^n\\
\end{displaymath}


\section{Integrals}


\subsection{Improper Integral Summary}

\def\arraystretch{3}
\begin{tabular}{lccr}
    Integral & $p \leq 1$ & $p > 1$ & Value\\
    $\displaystyle\int_{0}^{1} \frac{1}{x^p}$ & divergent & convergent & $\displaystyle\frac{1}{1-p}$\\
    $\displaystyle\int_{1}^{\infty} \frac{1}{x^p}$ & divergent & convergent & $\displaystyle\frac{1}{p-1}$\\
\end{tabular}


\subsection{Comparison Theorem}

If $f$ and $g$ are continuous and
$f(x) \geq g(x) \geq 0$ for $x \geq a$
(there is some $a$ where $f$ is now always larger than $g$)
then,\\
\\
If $\int_{a}^{\infty} f(x)dx$ is convergent then the ``smaller'' integral
$\int_{a}^{\infty} g(x)dx$ must be convergent too.\\
\\
If $\int_{a}^{\infty} g(x)dx$ is divergent then the ``larger'' integral
$\int_{a}^{\infty} f(x)dx$ must be divergent too.\\


\section{Sequences}


\subsection{Precise Limit Definition}

Say we have an arbitrary number $\epsilon > 0$ as a ``tolerance band'' from the limit $L$.
Assume the sequence converges.
There will be some integer $N$ where every $n > N$ holds $\left|a_n - L\right| < \epsilon$.
\\
\\
This allows subsequent terms in the sequence to oscillate around the limit $L$, so long as they remain in our tolerance band $\epsilon$.


\subsection{Convergence}

A sequence is convergent if:
\begin{itemize}
\item Its limit exists.
\item We can make $a_n$ closer and closer to $L$ by increasing $n$.
\end{itemize}


\subsection{Limit Theorems}

\begin{itemize}
\item If $\lim_{x\to\infty} f(x) = L$ and $f(n) = a_n$ when $n$ is an integer, then the sequence has the same limit.
    Essentially, if a function has the same value as the sequence for every integer, then its limit is the same.
\item Given an arbitrary value, there will be a number $N$ where every $a_n, n > N$ is larger than the arbitrary value, if the sequence diverges to infinity.
\item $\lim_{n\to\infty} a_n^p = \left[\lim_{n\to\infty} a_n\right]^p$ if $p > 0$ and $a_n > 0$
\item $\lim_{n\to\infty} \left|a_n\right| = 0$ then $\lim_{n\to\infty} a_n = 0$. If the limit of the absolute terms of the sequence is 0, then the limit of the terms is 0.
\item If the terms of a convergent sequence ($\lim a_n = L$) are applied to a continuous function, then the result is convergent too. $$\lim_{n\to\infty} f(a_n) = f(L)$$
\end{itemize}


\subsection{Squeeze Theorem}

If $a_n \leq b_n \leq c_n$ for $n \geq n_0$ and
$$\lim_{n\to\infty} a_n = \lim_{n\to\infty} c_n = L$$
then
$$\lim_{n\to\infty} b_n = L$$


\subsection{${r^n} sequences$}

Sequences defined as ${r^n}$ are convergent if $-1 < r \leq 1$.

\[ \lim_{n\to\infty} r^n =
    \begin{cases}
        0 & \quad \text{if } -1 < r < 1\\
        1 & \quad \text{if } r = 1\\
    \end{cases}
\]


\subsection{Monotonic Sequence Theorem}

Every bounded, monotonic sequence is convergent.


\section{Series}



\end{document}
