\documentclass[12pt]{article}
\usepackage[letterpaper, margin=0.5in]{geometry}
\usepackage{amsmath}
\DeclareMathSizes{12}{14}{10}{10}

\begin{document}


\section{Limits}


\subsection{$e$}

The function $e$ is defined as a continuous, differentiable function
$f(x)$ that satisfies $f'(x) = f(x)$ for all $x$ and $f(0) = 1$.\\
\\
\begin{displaymath}
    e = \displaystyle\lim_{n \to 0} \left(1 + n\right)^{\frac{1}{n}}
\end{displaymath}
\\
\begin{displaymath}
    e = \lim_{n \to \infty} \left(1 + \frac{1}{n}\right)^n\\
\end{displaymath}


\section{Integrals}


\subsection{Improper Integral Summary}

\def\arraystretch{3}
\begin{tabular}{lccr}
    Integral & $p \leq 1$ & $p > 1$ & Value\\
    $\displaystyle\int_{0}^{1} \frac{1}{x^p}$ & divergent & convergent & $\displaystyle\frac{1}{1-p}$\\
    $\displaystyle\int_{1}^{\infty} \frac{1}{x^p}$ & divergent & convergent & $\displaystyle\frac{1}{p-1}$\\
\end{tabular}


\subsection{Comparison Theorem}

If $f$ and $g$ are continuous and
$f(x) \geq g(x) \geq 0$ for $x \geq a$
(there is some $a$ where $f$ is now always larger than $g$)
then,\\
\\
If $\int_{a}^{\infty} f(x)dx$ is convergent then the ``smaller'' integral
$\int_{a}^{\infty} g(x)dx$ must be convergent too.\\
\\
If $\int_{a}^{\infty} g(x)dx$ is divergent then the ``larger'' integral
$\int_{a}^{\infty} f(x)dx$ must be divergent too.\\

\end{document}
