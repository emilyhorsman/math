\documentclass[12pt]{article}
\renewcommand{\baselinestretch}{1.2}
\usepackage[letterpaper, margin=0.5in]{geometry}
\usepackage{amsmath}
\DeclareMathSizes{12}{13}{9.5}{9}

\begin{document}


\section{Trigonometric Identities}


\subsection{Double-Angle Formulas}

You can use the formula for $\cos(2x)$ with the identity $\sin^2x + \cos^2x = 1$ to produce other useful formulas.\\
\begin{align*}
    \sin(2x) &= 2\sin(x)\cos(x)\\
    \cos(2x) &= \cos^2(x) - \sin^2(x)&**\\
    &= \cos^2(x) - (1 - \cos^2(x))\\
    &= 2\cos^2(x) - 1&**\\
    \cos(2x) &= \cos^2(x) - \sin^2(x)\\
    &= (1 - \sin^2(x)) - \sin^2(x)\\
    &= 1 - 2\sin^2(x)&**\\
\end{align*}


\subsection{Half-Angle Formulas}

You can then use the double-angle formulas to derive the following.\\
These are the ones most useful for integral calculus.\\
Memorizing the original double-angle formulas allows one to derive these easily.

\begin{align*}
    \cos^2(x) &= \frac{1+\cos(2x)}{2}\\
    \sin^2(x) &= \frac{1-\cos(2x)}{2}
\end{align*}


\section{Limits}


\subsection{$e$}

The function $e$ is defined as a continuous, differentiable function
$f(x)$ that satisfies $f'(x) = f(x)$ for all $x$ and $f(0) = 1$.\\
\\
\begin{displaymath}
    e = \displaystyle\lim_{n \to 0} \left(1 + n\right)^{\frac{1}{n}}
\end{displaymath}
\\
\begin{displaymath}
    e = \lim_{n \to \infty} \left(1 + \frac{1}{n}\right)^n\\
\end{displaymath}


\section{Integrals}


\subsection{Improper Integral Summary}

\def\arraystretch{3}
\begin{tabular}{lccr}
    Integral & $p \leq 1$ & $p > 1$ & Value\\
    $\displaystyle\int_{0}^{1} \frac{1}{x^p}$ & divergent & convergent & $\displaystyle\frac{1}{1-p}$\\
    $\displaystyle\int_{1}^{\infty} \frac{1}{x^p}$ & divergent & convergent & $\displaystyle\frac{1}{p-1}$\\
\end{tabular}


\subsection{Comparison Theorem}

If $f$ and $g$ are continuous and
$f(x) \geq g(x) \geq 0$ for $x \geq a$
(there is some $a$ where $f$ is now always larger than $g$)
then,\\
\\
If $\int_{a}^{\infty} f(x)dx$ is convergent then the ``smaller'' integral
$\int_{a}^{\infty} g(x)dx$ must be convergent too.\\
\\
If $\int_{a}^{\infty} g(x)dx$ is divergent then the ``larger'' integral
$\int_{a}^{\infty} f(x)dx$ must be divergent too.\\


\section{Sequences}


\subsection{Precise Limit Definition}

Say we have an arbitrary number $\epsilon > 0$ as a ``tolerance band'' from the limit $L$.
Assume the sequence converges.
There will be some integer $N$ where every $n > N$ holds $\left|a_n - L\right| < \epsilon$.
\\
\\
This allows subsequent terms in the sequence to oscillate around the limit $L$, so long as they remain in our tolerance band $\epsilon$.


\subsection{Convergence}

A sequence is convergent if:
\begin{itemize}
\item Its limit exists.
\item We can make $a_n$ closer and closer to $L$ by increasing $n$.
\end{itemize}


\subsection{Limit Theorems}

\begin{itemize}
\item If $\lim_{x\to\infty} f(x) = L$ and $f(n) = a_n$ when $n$ is an integer, then the sequence has the same limit.
    Essentially, if a function has the same value as the sequence for every integer, then its limit is the same.
\item Given an arbitrary value, there will be a number $N$ where every $a_n, n > N$ is larger than the arbitrary value, if the sequence diverges to infinity.
\item $\lim_{n\to\infty} a_n^p = \left[\lim_{n\to\infty} a_n\right]^p$ if $p > 0$ and $a_n > 0$
\item $\lim_{n\to\infty} \left|a_n\right| = 0$ then $\lim_{n\to\infty} a_n = 0$. If the limit of the absolute terms of the sequence is 0, then the limit of the terms is 0.
\item If the terms of a convergent sequence ($\lim a_n = L$) are applied to a continuous function, then the result is convergent too. $$\lim_{n\to\infty} f(a_n) = f(L)$$
\end{itemize}


\subsection{Squeeze Theorem}

If $a_n \leq b_n \leq c_n$ for $n \geq n_0$ and
$$\lim_{n\to\infty} a_n = \lim_{n\to\infty} c_n = L$$
then
$$\lim_{n\to\infty} b_n = L$$


\subsection{${r^n}$ sequences}

Sequences defined as ${r^n}$ are convergent if $-1 < r \leq 1$.

\[ \lim_{n\to\infty} r^n =
    \begin{cases}
        0 & \quad \text{if } -1 < r < 1\\
        1 & \quad \text{if } r = 1\\
    \end{cases}
\]


\subsection{Monotonic Sequence Theorem}

Every bounded, monotonic sequence is convergent.


\section{Series}


\subsection{Definition}

A series is simply the sum of terms in a sequence.\\
An infinite series (often simply just called a ``sum'' or ``series'') is what we get when we sum an infinite number of terms in a sequence.\\
A partial sum is what we get when we sum a finite number of terms in a sequence, $s_3$ for ${a_n}$ is the sum of $a_1, a_2,\text{ and } a_3$.\\
A series $s_n$ forms its own sequence ${s_n}$.\\
As we increase $n$ in ${s_n}$ we get closer and closer to the limit---the infinite sum---of the series.\\
\\
\begin{displaymath}
s_n = \sum_{i=1}^{n} a_i = a_1 + a_2 + \dots + a_n
\end{displaymath}
\\
\begin{displaymath}
s = \sum_{n=1}^{\infty} a_n = a_1 + a_2 + \dots + a_n + \dots
\end{displaymath}


\subsection{Example}

Take the sequence $a_n = \frac{1}{2^n}$.\\
This gives us $a_1 = \frac{1}{2}, a_2 = \frac{1}{4},\text{ etc}\dots$.\\
$s_2$ would then be $a_1 + a_2 = \frac{3}{4}$.\\
This forms a sequence from the series, ${\frac{1}{2}, \frac{3}{4}, \frac{7}{8}, \dots}$.\\
This sequence converges on 1 the more terms we add.\\
This is the infinite sum.\\
\begin{displaymath}
\sum_{n=1}^{\infty} \frac{1}{2^n} = \frac{1}{2} + \frac{3}{4} + \dots + \frac{1}{2^n} + \dots = 1
\end{displaymath}
\\
The sum of a series is $s = \lim_{n\to\infty} s_n$.\\
The series will be divergent---and not have a sum---if the sequence ${s_n}$ diverges.


\subsection{Geometric Series}

A geometric series occurs when each term of the sequence is multiplied by the preceding one by a common ratio.\\
\begin{displaymath}
a \neq 0 \quad \sum_{n=1}^{\infty} ar^{n-1} = a + ar + ar^2 + ar^3 + \dots + ar^{n-1} + \dots
\end{displaymath}
\\
This series is convergent if $\left|r\right| < 1$.\\
The partial sum is defined by the following.\\
\begin{displaymath}
s_n = \frac{a(1 - r^n)}{1 - r}
\end{displaymath}
The sum of a convergent geometric series is defined by the following.\\
\begin{displaymath}
\sum_{n=1}^{\infty} ar^{n-1} = \frac{a}{1-r}
\end{displaymath}


\subsection{Test for Divergence}

If the series is convergent then the limit of $a_n$ will be 0.\\
However, we cannot conclude that a series if convergent just because $a_n$ has a limit of 0.\\
\begin{displaymath}
\text{If }\sum a_n\text{ converges, then }\lim_{n\to\infty} a_n = 0
\end{displaymath}
\\
Even though we cannot make conclusions about the series being convergent, we can check if the series is divergent.\\
\begin{displaymath}
\sum a_n\text{ divergent if }\lim_{n\to\infty} a_n \neq 0\text{ or the limit does not exist.}
\end{displaymath}


\subsection{Integral Test}

If $f$ is continuous, positive, and decreasing on $[1, \infty)$ then let $a_n = f(n)$.\\
\begin{displaymath}
\text{If }\int_{1}^{\infty} f(x)dx\text{ is convergent, then }\sum_{n=1}^{\infty} a_n\text{ is convergent.}
\end{displaymath}
\\
\begin{displaymath}
\text{If }\int_{1}^{\infty} f(x)dx\text{ is divergent, then }\sum_{n=1}^{\infty} a_n\text{ is divergent.}
\end{displaymath}


\subsection{p-series Test}

\begin{displaymath}
\sum_{n=1}^{\infty} \frac{1}{n^p}\text{ is convergent if }p > 1\text{ and divergent if }p \leq 1\text{.}
\end{displaymath}


\subsection{Error with the Integral Test}

Estimating an infinite sum with a finite number of terms yields a remainder $R_n = s - s_n$.\\
This remainder is our error.\\
\begin{displaymath}
\int_{n+1}^{\infty} f(x)dx \leq R_n \leq \int_{n}^{\infty} f(x)dx
\end{displaymath}
\\
\begin{displaymath}
s_n + \int_{n+1}^{\infty} f(x)dx \leq s \leq s_n + \int_{n}^{\infty} f(x)dx
\end{displaymath}


\subsection{Comparison Tests}

If $a_n \leq b_n$ for all $n$, then $\sum a_n$ will be convergent if $\sum b_n$ is convergent.\\
If $a_n \geq b_n$ for all $n$, then $\sum a_n$ will be divergent if $\sum b_n$ is divergent.\\
Essentially, if a bigger series is convergent, then smaller series must be as well.\\
If a smaller series is divergent, then larger series must be as well.


\subsection{Limit Comparison Test ($c > 0$)}

\begin{displaymath}
\lim_{n\to\infty} \frac{a_n}{b_n} = c > 0
\end{displaymath}
\\
If this holds ($c > 0$) then either both $\sum a_n$ and $\sum b_n$ converge, or they both diverge.


\subsection{Alternating Series Test}

An alternating series is one defined by the following.\\
\begin{displaymath}
\sum (-1)^{n-1} b_n\quad b_n > 0
\end{displaymath}
If the following are satisfied then the series will be convergent.
\begin{itemize}
\item $b_{n+1} \leq b_n$ for all $n$
\item $\lim_{n\to\infty} b_n = 0$
\end{itemize}


\subsection{Alternating Series Estimation Theorem}

If the Alternating Series Test is satisfied then the following holds.
\\
\begin{displaymath}
\left|R_n\right| = \left|s - s_n\right| \leq b_{n+1}
\end{displaymath}
\\
This lets us find a desired error by simply computing a value in the sequence.
\\
How many terms are needed to estimate the sum $\sum \frac{(-1)^n}{n^6}$ such that $|\text{error}| < 0.00005$?\\
\begin{align*}
\left|\text{error}\right| \leq b_{n+1} &< \frac{1}{20000}\\
(n+1)^6 &> 20000\\
(5+1)^6 &> 20000\\
    6^6 &> 20000\\
    b_6 &> 20000\\
\end{align*}
The sixth term is less than the desired error.\\
Adding the sixth term will not yield more than our desired accuracy.\\
This means we need five terms to yield a sum with the desired accuracy.


\subsection{Error}

$$0.00005 = \frac{1}{20,000}$$
$$0.00001 = \frac{1}{100,000}$$
$$0.01 = \frac{1}{100}$$
$$0.02 = \frac{1}{50}$$
$$0.05 = \frac{1}{20}$$


\subsection{Absolute and Conditional Convergence}

A series is absolutely convergent if $\sum \left|a_n\right|$ is convergent.\\
A series is convergent if it is absolutely convergent.\\
A series is conditionally convergent if $\sum a_n$ is convergent but not $\sum \left|a_n\right|$.\\


\subsection{Ratio and Ratio Test}

Take a series $\sum a_n$.
\\
\begin{displaymath}
    \lim_{n\to\infty} \left|\frac{a_{n+1}}{a_n}\right| = L\quad\text{ or }\quad
    \lim_{n\to\infty} \sqrt[n]{\left|a_n\right|} = L
\end{displaymath}
\\
If $L < 1$ then the series is absolutely convergent.\\
If $L > 1\text{ or }L = \infty$ then the series is divergent.\\
If $L = 1$ then the nothing about the series can be concluded.



\end{document}
