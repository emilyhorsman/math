\documentclass[12pt]{article}
\renewcommand{\baselinestretch}{1.2}
\usepackage[letterpaper, margin=0.5in]{geometry}
\usepackage{amsmath}
\DeclareMathSizes{12}{13}{9.5}{9}

\begin{document}


\section{Definitions}


\subsection{Trivial Solution}

A zero vector. If $Ax = 0$ has only the trivial solution then $x$ must be something like,
$$
\begin{bmatrix}
    x_1\\
    x_2\\
    \vdots\\
    x_r
\end{bmatrix}
=
\begin{bmatrix}
    0\\
    0\\
    \vdots\\
    0
\end{bmatrix}
$$


\section{Equivalence Theorem}

If $A$ is an $n \times n$ matrix, then the following statements are equivalent. That is, if one is true, the rest is true, as they are logically equivalent.
\begin{itemize}
    \item $A$ is invertible.
    \item $Ax = 0$ has only the trivial solution.
    \item The reduced row echelon form of $A$ is $I_n$.
    \item $A$ is expressible as a product of elementary matrices. $A = E_nE_{n-1}\dots E_1I_n$.
    \item $Ax = b$ has exactly one solution for every $n \times 1$ matrix $b$.
    \item $det(A) \neq 0$.
    \item $\lambda = 0$ is not an eigenvalue of $A$.
\end{itemize}


\end{document}
