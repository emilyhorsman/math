\documentclass[12pt]{article}
\renewcommand{\baselinestretch}{1.2}
\usepackage[letterpaper, margin=0.5in]{geometry}
\usepackage{amsmath}
\DeclareMathSizes{12}{13}{9.5}{9}
\DeclareMathOperator{\adj}{adj}

\begin{document}


\section{Definitions}


\subsection{Trivial Solution}

A zero vector. If $Ax = 0$ has only the trivial solution then $x$ must be something like,
$$
\begin{bmatrix}
    x_1\\
    x_2\\
    \vdots\\
    x_r
\end{bmatrix}
=
\begin{bmatrix}
    0\\
    0\\
    \vdots\\
    0
\end{bmatrix}
$$


\subsection{Symmetric Matrix}

A square matrix $A$ where $A = A^T$. Thus, $(A)_{ij} = (A)_{ji}$.
$$
\begin{bmatrix}1 & 9\\9 & 2\end{bmatrix}
$$


\subsection{Skew-symmetric Matrix}

A square matrix $A$ where $A^T = -A$.\\
All the main diagonal entries must be 0.
\begin{align*}
    -(A_{ij}) &= (A^T)_{ij}\\
    -(A_{ij}) &= A_{ji}\\
    -(A_{ii}) &= A_{ii}&\text{On the diagonal, }i = j\\
    A_{ii} &= 0&\text{0 is the only value that will hold}
\end{align*}


\section{Equivalence Theorem}

If $A$ is an $n \times n$ matrix, then the following statements are equivalent. That is, if one is true, the rest is true, as they are logically equivalent.
\begin{itemize}
    \item $A$ is invertible.
    \item $Ax = 0$ has only the trivial solution.
    \item The reduced row echelon form of $A$ is $I_n$.
    \item $A$ is expressible as a product of elementary matrices. $A = E_nE_{n-1}\dots E_1I_n$.
    \item $Ax = b$ has exactly one solution for every $n \times 1$ matrix $b$.
    \item $det(A) \neq 0$.
    \item $\lambda = 0$ is not an eigenvalue of $A$.
\end{itemize}


\section{Determinant Properties}


\subsection{Adjoint Matrices}

We know the following:
\begin{align*}
    A\adj(A) &= \det(A)I\\
    \adj(A) &= A^{-1}\det(A)I
\end{align*}
We can then find the determinant of the adjoint of a matrix in terms of the determinant of the original matrix.
\begin{align*}
    A &= \adj(B)\\
    A &= B^{-1}\det(B)I\\
    \det(A) &= \det(B^{-1}) \det(\det(B)) \det(I)\\
    \det(A) &= \det(B)^{-1} \det(B)^n 1\\
    \det(A) &= \det(B)^{n-1}
\end{align*}


\end{document}

\iffalse
Ax - \lambda x = 0
Infinitely many solutions means det(A - \lambda I) = 0
Exactly one solution means det(A - \lambda I) \neq 0 (this means the matrix is invertible, which means it has exactly one solution)
The bases is all the distinct eigenvectors
\fi
